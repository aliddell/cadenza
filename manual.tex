
\documentclass[11pt]{report}
\usepackage{amsmath,amssymb,amsfonts}
\usepackage{hyperref}
\usepackage{epsfig}
\usepackage{color}

\topmargin 0.0in
\textheight 8.5in
\textwidth 6.5in
\oddsidemargin 0in
\evensidemargin 0in
\parskip 1mm
\setlength{\parindent}{10mm}





\begin{document}

\newcommand{\bC}{{\mathbb C}}
\newcommand{\dhat}{{\hat d}}
\newcommand{\inputF}{{\tt input}}
\newcommand{\start}{{\tt start}}
\newcommand{\testPts}{{\tt test\_points}}
\newcommand{\outputF}{{\tt output}}
\newcommand{\localDim}{{\tt local\_dim}}
\newcommand{\blueharvest}{{Blue Harvest}}
\newcommand{\blueharvestS}{{\blueharvest~}}



\def\thesection{\arabic{section}}

%Front matter:
\title{\blueharvest}
\author{Alan C. Liddell, Jr.}
\maketitle

%\tableofcontents

\chapter*{\blueharvest}

\section{Introduction to \blueharvest}\label{CHAP:intro}

The program \blueharvestS by Jonathan D. Hauenstein and Frank Sottile
implements algorithms based on Smale's $\alpha$-theory to
certify solutions to polynomial and polynomial-exponential systems.
This manual provides detailed instructions on how to use \blueharvestS
while \cite{HS10,HL11} provides more information regarding the mathematical
theory underlying \blueharvest.

\section{Compiling \blueharvest}\label{CHAP:compile}

The program \blueharvestS is written in C and uses the GMP\cite{GMP} and MPFR\cite{MPFR} libraries
to perform rational and arbitrary floating point arithmetic.
To compile \blueharvest, the user needs to verify the settings in {\tt Makefile}
associated with the C compiler and the location of these libraries.
The following is an example of the first three lines in {\tt Makefile}
which specifies using {\tt gcc} along with the
locations of the GMP and MPFR installation directories.

$$\begin{array}{l}
\hbox{\begin{color}{blue}COMP\end{color}=gcc} \\
\hbox{\begin{color}{blue}GMP\end{color}=/home/GMP\_4\_3\_2/} \\
\hbox{\begin{color}{blue}MPFR\end{color}=/home/MPFR\_2\_4\_2/}
\end{array}$$

The following is the next two lines in {\tt Makefile}
which specifies to the C compiler which libraries to link to
and location of the header files.

$$\begin{array}{l}
\hbox{\begin{color}{blue}LIB\end{color}=-lm -L\begin{color}{blue}\$(MPFR)\end{color}/lib/ -lmpfr -L\begin{color}{blue}\$(GMP)\end{color}/lib/ -lgmp} \\
\hbox{\begin{color}{blue}INC\end{color}=-I\begin{color}{blue}\$(MPFR)\end{color}/include -I\begin{color}{blue}\$(GMP)\end{color}/include}
\end{array}$$

When using {\tt gcc}, for example, if the proper static libraries were created,
adding the ``-static'' option on the \begin{color}{blue}LIB\end{color} line, displayed below,
will create a statically linked executable.

$$\begin{array}{l}
\hbox{\begin{color}{blue}LIB\end{color}=-static -lm -L\begin{color}{blue}\$(MPFR)\end{color}/lib/ -lmpfr -L\begin{color}{blue}\$(GMP)\end{color}/lib/ -lgmp}
\end{array}$$

Once {\tt Makefile} is setup, simply run `make' to compile \blueharvest.

\section{Using \blueharvest}\label{CHAP:use}

To use \blueharvest, the user needs to create at least two files which specify
the polynomial or polynomial-exponential system and the points to test.  
An optional third file can be used to adjust the configuration settings.  
See Appendix~\ref{CHAP:configs} for a detailed description of each configuration setting.

\subsection{Polynomial systems}\label{SEC:PolySys}

A polynomial system is entered into a file
by listing the monomials and the coefficients appearing in each polynomial.
The system must be square or overdetermined and each
coefficient must be a complex \begin{color}{red}rational\end{color} number.

The first line of the file lists both the number of variables and the number of
polynomials for the polynomial system.  Then, the file contains a block for each polynomial
which contains the number of terms followed by the degrees of
each variable in the monomial and the real and imaginary parts of its coefficient.
For example, if the polynomial system depends upon three variables, say $x$, $y$, and $z$,
the term
$$\left(\frac{1}{2} + 3i\right) xy^2z$$
would be written as
$$\begin{array}{ccccc}
1 & 2 & 1 & 1/2 & 3
\end{array}$$
since the degrees of $x$, $y$, and $z$ in $xy^2z$ are 1, 2, and 1, respectively,
and the coefficient has real part $\displaystyle\frac{1}{2}$ and imaginary part $3$.

For a complete example, consider $f(x,y,z) = \left[\begin{array}{c} x^2 - 3xy + z^4 \\
\left(\frac{1}{2} + 3i\right) xy^2z - 9 z + 2x + 7\\ z^{10} - i \end{array}\right]$.
The polynomial system file for $f$ is as follows.

\begin{table}[h!]
\centering
$\begin{array}{|lllll|}
\hline
3 & 3 & & & \\
& & & & \\
3 & & & & \\
2 & 0 & 0 & 1  & 0 \\
1 & 1 & 0 & -3 & 0 \\
0 & 0 & 4 & 1  & 0 \\
4 & & & & \\
1 & 2 & 1 & 1/2 & 3 \\
0 & 0 & 1 & -9 & 0 \\
1 & 0 & 0 & 2 & 0 \\
0 & 0 & 0 & 7 & 0 \\
2 & & & & \\
0 & 0 & 10 & 1 & 0 \\
0 & 0 & 0 & 0 & -1 \\
\hline
\end{array}$
\caption{Example of a polynomial system file for $f$}\label{Tab:PolySys}
\end{table}

\subsection{Polynomial-exponential systems}\label{SEC:PolyExpSys}

Starting with version 1.2, alphaCertified implements the algorithms
of \cite{HL11} for certifying solutions to square systems of polynomial-exponential
functions.  The polynomial-exponential system must be of the form
\[
\left[\begin{array}{cc}
P_i(x_1,\dots,x_n,y_1,\dots,z_m), & i = 1,\dots,n \\
y_i - g_i(\beta_i x_{\sigma_i}), & i = 1,\dots,m
\end{array}\right]
\]
where each $P_i$ is a polynomial with complex \begin{color}{red}rational\end{color} coefficients,
$\beta_i$ is a complex \begin{color}{red}rational\end{color} number, and
$g_i(z)$ is either $\exp(z)$, $\sin(z)$, $\cos(z)$, $\sinh(z)$, or $\cosh(z)$.
Moreover, polynomial-exponential certification must use 
floating point arithmetic (see Appendix~\ref{CHAP:configs} for more details).  

The first line of the file lists both the number of variables, e.g., $n + m$,
and the number of polynomials in the polynomial-exponential system, e.g., $n$.  
Then, the file contains a block for each polynomial 
following the structure described in Section~\ref{SEC:PolySys}.  
The final block contains a line for each additional function which lists
the integer $\sigma_i$, a string describing the function $g_i$, and the
real and imaginary parts of $\beta_i$.  The following table lists the
strings for the possible functions.

\begin{table}[h!]
\centering
$\begin{array}{c|c}
\mbox{function} & \mbox{string} \\ 
\hline
\exp(z) & X \\
\sin(z) & S \\
\cos(z) & C \\
\sinh(z) & SH \\
\cosh(z) & CH
\end{array}$
\caption{Functions and strings}\label{Tab:Strings}
\end{table}

For a complete example, consider $f(x,y,z) = \left[\begin{array}{c} x^2 - 3xy + z^4 \\
\left(\frac{1}{2} + 3i\right) xy^2z - 9 z + 2x + 7\\ z - \cos((3 - 4i)y) \end{array}\right]$.
The polynomial-exponential system file for $f$ is as follows.

\begin{table}[h!]
\centering
$\begin{array}{|lllll|}
\hline
3 & 2 & & & \\
& & & & \\
3 & & & & \\
2 & 0 & 0 & 1  & 0 \\
1 & 1 & 0 & -3 & 0 \\
0 & 0 & 4 & 1  & 0 \\
4 & & & & \\
1 & 2 & 1 & 1/2 & 3 \\
0 & 0 & 1 & -9 & 0 \\
1 & 0 & 0 & 2 & 0 \\
0 & 0 & 0 & 7 & 0 \\
  &   &   &   &   \\
2 & C & 3 & -4 &  \\
\hline
\end{array}$
\caption{Example of a polynomial-exponential system file for $f$}\label{Tab:PolyExpSys}
\end{table}

\subsection{Input points}\label{SEC:Points}

The points are read into \blueharvestS using the requested arithmetic type (see Appendix~\ref{CHAP:configs}
for more details).  In particular, if \blueharvestS is configured to use rational arithmetic,
the real and imaginary parts of the coordinates for each point must be a rational number.
Likewise, if \blueharvestS is configured to use $P$-bit floating point arithmetic, the
the real and imaginary parts of the coordinates for each point
must be a floating point number and are read into \blueharvestS using $P$-bit precision.
When using floating point arithmetic, many of the output files
generated by Bertini \cite{Bertini} can be used for input into \blueharvest.

The first line of the file lists the number of points in the file.
This is followed by a block for each point which contains the real and imaginary
parts of each coordinate.
For example, the following presents
$$S = \left\{\left[\begin{array}{c} \frac{2}{3}+2i \\ -7 \\ 3i \end{array}\right],
\left[\begin{array}{c} \frac{1}{8} \\ -7+2i \\ -1-\frac{2}{5}i \end{array}\right]\right\}$$
using rational and 10-digit floating point representation.

\begin{table}[h!]
\centering
$\begin{array}{|ll|}
\hline
2 & \\
 & \\
2/3 & 2\\
-7 & 0 \\
0 & 3 \\
 & \\
1/8 & 0 \\
-7 & 2 \\
-1 & -2/5 \\
\hline
\end{array}$
\caption{Example using rational numbers for $S$}\label{Tab:RatPts}
\end{table}

\begin{table}[h!]
\centering
$\begin{array}{|ll|}
\hline
2 & \\
 & \\
0.6666666667 & 2\\
-7 & 0 \\
0 & 3 \\
 & \\
0.125 & 0 \\
-7 & 2 \\
-1 & -0.4 \\
\hline
\end{array}$
\caption{Example using floating point numbers for $S$}\label{Tab:FloatPts}
\end{table}

\subsection{Configuration settings}\label{SEC:Settings}

To adjust the configuration settings, the user needs to create
a file listing the configurations along with the requested values.
See Appendix~\ref{CHAP:configs} for each configurable setting
and its acceptable values.

For example, the following presents settings that would
only certify approximate solutions using 192-bit floating point precision.

\begin{table}[h!]
\centering
$\begin{array}{|l|}
\hline
\hbox{ALGORITHM: 0;} \\
\hbox{ARITHMETICTYPE: 1;} \\
\hbox{PRECISION: 192;} \\
\hline
\end{array}$
\caption{Example of configuration settings}\label{Tab:Configs}
\end{table}

\subsection{Running \blueharvest}\label{SEC:Run}

The command line arguments for \blueharvestS specify the names
of the files to use.  The default names for the corresponding
files are {\tt polynomialSystem}, {\tt testPoints}, and {\tt settings}.

The command line arguments correspond to the names of these files in order.
For example, if there are two arguments, the first is the name of the
system file and the second is the name of the file containing the points to test.
Default names will be used when less than three command line arguments are used.
Errors will be returned if either the system or the
test points file do not exist.  If the configuration settings file does not exist,
the default settings will be used.

For example, if there is no file named {\tt settings} in the current folder,
the following Linux command runs \blueharvestS using the default settings
on the system file named {\tt polySys}
and the test points file named {\tt points}.
$$\gg \hbox{./alphaCertified polySys points}$$

\section{Output of \blueharvest}\label{CHAP:output}

The output of \blueharvestS is a collection of files that are dependent
upon the configuration settings and the polynomial system as well as an onscreen
summary.  The following lists the possible files created by \blueharvest,
their format, and the settings needed to create them.  See Appendix~\ref{CHAP:configs}
for more details regarding the configurations settings.

\begin{itemize}

\item {\tt approxSolns}

This file, in the format of the input points files described in Section~\ref{SEC:Points},
lists the points which are certifiably approximate solutions.  This file is always
created.

\item {\tt constantValues}

This file lists an upper bound of $\alpha$, an approximation of $\beta$, and an upper bound
of $\gamma$ for each point (see \cite{HS10} for details on how these values are computed).
These values are always printed using a 16-digit floating point representation.  The format is similar
to that presented in Table~\ref{Tab:FloatPts}.  In particular, the first line lists the number
of points and then a block for each point which lists the computed values for $\alpha$, $\beta$,
and $\gamma$ on separate lines.  This file is always created.

\item {\tt distinctSolns}

This file, in the format of the input points files described in Section~\ref{SEC:Points},
lists the points which are certifiably approximate solutions that correspond to distinct
solutions.  This file is created when ALGORITHM is at least 1.

\item {\tt isApproxSoln}

This file lists boolean values (0 or 1) describing whether each point has been
certified to be an approximate solution.   The first line lists the number of points
and then the boolean value is listed for each point.  Note that a value of 0 means
that \blueharvestS was not able to certify that it was an approximate solution.
This file is always created.

\item {\tt isDistinctSoln}

This file, in the same format as {\tt isApproxSoln}, lists values describing whether
each point corresponds to a distinct solution.  A value of $-2$ means that the
point was not able to be certified as an approximate solution.  A value of $-1$
means that the point corresponds to a solution that is distinct from the
certifiably approximate solutions that precede it in the list of points.
A nonnegative value, say $j$, means that this point and the $j^{th}$ point
correspond to the same solution, where the points are numbered staring with 0.
This file is created when ALGORITHM is at least 1.

\item {\tt isRealSoln}

This file, in the same format as {\tt isApproxSoln}, lists values describing whether
each point corresponds to a real solution.  A value of $-2$ means that the point
was not able to be certified as an approximate solution.  Otherwise,
the value is a boolean value (0 or 1) describing if the point corresponds to a real solution.
This file is created when ALGORITHM
is 2 and the polynomial system is real.

\item {\tt nonrealDistinctSolns}

This file, in the format of the input points files described in Section~\ref{SEC:Points},
lists the points which are certifiably approximate solutions that correspond to distinct
nonreal solutions.  This file is created when ALGORITHM is 2 and the polynomial system is real.

\item {\tt realDistinctSolns}

This file, in the format of the input points files described in Section~\ref{SEC:Points},
lists the points which are certifiably approximate solutions that correspond to distinct
real solutions.  This file is created when ALGORITHM is 2 and the polynomial system is real.

\item {\tt redundantSolns}

This file, in the format of the input points files described in Section~\ref{SEC:Points},
lists the points which are certifiably approximate solutions and correspond
to the same solution as another certifiably approximate solution that precedes
it in the list of points.  This file is created when ALGORITHM is at least 1.

\item {\tt refinedPoints}

This file, in the format of the input points files described in Section~\ref{SEC:Points},
lists the most accurate internally computed approximation of the corresponding solution.
If REFINEDIGITS is positive, say $\tau$, then, for each certifiable approximate solution,
the point listed in this file is within $10^{-\tau}$ of the corresponding solution.
This file is always created.

\item {\tt summary}

This file is a human-readable summary for each point.  The first part of this file
reprints the onscreen summary of the results.  This is followed by a block for each
point which lists the point, the results for that point, and the
computed values of $\alpha$, $\beta$, and $\gamma$ for both the original point and its
corresponding point printed in {\tt refinedSolns} (see \cite{HS10} for details on how these values are computed).
The last part of this file contains configuration settings and version information for \blueharvest.
This file is always created.

\item {\tt unknownPoints}

This file, in the format of the input points files described in Section~\ref{SEC:Points},
lists the points which can not be certified as approximate solutions.
This file is always created.

\end{itemize}

\section{Performing Newton iterations using \blueharvest}\label{CHAP:Newton}

When using \blueharvestS for certifying solutions, the certified approximate
solutions can be refined using Newton's method to any given accuracy
using the REFINEDIGITS configuration setting.
Instead of only refining the certified approximate solutions, \blueharvestS
can also be used to perform Newton iterations on all of the input points
using the NEWTONONLY and NUMITERATIONS configuration settings.
See Appendix~\ref{CHAP:configs} for more details regarding
these configuration settings.

Note that if floating point arithmetic is begin used, i.e., ARITHMETICTYPE is 1,
the internal working precision is automatically increased during each iteration.

\section{Maple interface for \blueharvest}\label{CHAP:maple}

The Maple interface for \blueharvestS can be used to construct
the input files, run \blueharvest, and read in output files.
After updating {\tt libname} to include the folder where
the \blueharvestS Maple interface is located, it can be loaded
with the following command:
\[
\begin{color}{red}{\bf >}~~with(alphaCertifiedMaple);\end{color}
\]
yielding the following output:
\[
\begin{color}{blue}
\begin{array}{l}
[alphaCertified, alphaCertifiedExp, defaultExpSettings, defaultSettings, loadOutput, \\
~~~~~~printPoints, printPolyExpSystem, printPolynomialSystem, printSettings]\end{array}\end{color}
\]
The following describes how to use these nine procedures.

\subsection{alphaCertified}\label{Sec:alphaM}

The Maple procedure alphaCertified constructs the input files
for a polynomial system, runs \blueharvest, and loads output data.

\[
\hbox{PointsData := alphaCertified(alphaPath, Polys, Vars, Points, Settings);}
\]

\begin{itemize}

\item {\tt alphaPath}

A string which is the path to an \blueharvestS executable file.

\item {\tt Polys}

A list containing the polynomial system.

\item {\tt Vars}

A list containing the variables of the polynomial system.

\item {\tt Points}

A two-dimensional list containing the points to test.

\item {\tt Settings (optional)}

A Maple record containing the configuration settings.
See Section~\ref{Sec:SettingsM} for more details constructing
this record.

\item {\tt PointsData}

A Maple record containing output data from \blueharvest.
The fields in this record are:
\begin{itemize}
 \item {\tt alpha}

A list containing an upper bound of $\alpha$ at each point.

 \item {\tt beta}

A list containing an approximation of $\beta$ at each point.

 \item {\tt gamma}

A list containing an upper bound of $\gamma$ at each point.

 \item {\tt refinedPts}

A two-dimensional list containing the refined points.

 \item {\tt isApproxSoln}

A list that describes if each point is a certifiable approximate solution.
Each value is either ``Unknown'', ``Yes'', or ``No''.

 \item {\tt isDistinctSoln}

A list that describes if each point corresponds to a distinct solution.
Each value is either ``Unknown'', ``Yes'', or ``No''.

 \item {\tt isRealSoln}

A list that describes if each point corresponds to a real solution.
Each value is either ``Unknown'', ``Yes'', or ``No''.

\end{itemize}
\end{itemize}

\subsubsection{Example}

\[
\begin{color}{red}\begin{array}{l}
> \hbox{alphaPath := ``./alphaCertified'';} \\
> \hbox{Polys := [$x*y^2 + x, x*y - y$];} \\
> \hbox{Vars := [$x,y$];} \\
> \hbox{Points := [[0,0],[1,1+I],[1,-I]];} \\
> \hbox{PointsData := alphaCertified(alphaPath, Polys, Vars, Points);}
\end{array}\end{color}
\]

\subsection{alphaCertifiedExp}\label{Sec:alphaExpM}

The Maple procedure alphaCertified constructs the input files
for a polynomial-exponential system, runs \blueharvest, and loads output data.

\[
\hbox{PointsData := alphaCertifiedExp(alphaPath, Funcs, Vars, Points, Settings);}
\]

\begin{itemize}

\item {\tt alphaPath}

A string which is the path to an \blueharvestS executable file.

\item {\tt Funcs}

A list containing the polynomial-exponential system.

\item {\tt Vars}

A list containing the variables of the polynomial-exponential system.

\item {\tt Points}

A two-dimensional list containing the points to test.

\item {\tt Settings (optional)}

A Maple record containing the configuration settings.
See Section~\ref{Sec:SettingsM} for more details constructing
this record.

\item {\tt PointsData}

A Maple record containing output data from \blueharvest.
The fields in this record are:
\begin{itemize}
 \item {\tt alpha}

A list containing an upper bound of $\alpha$ at each point.

 \item {\tt beta}

A list containing an approximation of $\beta$ at each point.

 \item {\tt gamma}

A list containing an upper bound of $\gamma$ at each point.

 \item {\tt refinedPts}

A two-dimensional list containing the refined points.

 \item {\tt isApproxSoln}

A list that describes if each point is a certifiable approximate solution.
Each value is either ``Unknown'', ``Yes'', or ``No''.

 \item {\tt isDistinctSoln}

A list that describes if each point corresponds to a distinct solution.
Each value is either ``Unknown'', ``Yes'', or ``No''.

 \item {\tt isRealSoln}

A list that describes if each point corresponds to a real solution.
Each value is either ``Unknown'', ``Yes'', or ``No''.

\end{itemize}
\end{itemize}

\subsubsection{Example}

\[
\begin{color}{red}\begin{array}{l}
> \hbox{alphaPath := ``./alphaCertified'';} \\
> \hbox{Funcs := [$x*y^2 + x, y - exp(3*x)$];} \\
> \hbox{Vars := [$x,y$];} \\
> \hbox{Points := [[0,1],[2,1+I]];} \\
> \hbox{PointsData := alphaCertifiedExp(alphaPath, Funcs, Vars, Points);}
\end{array}\end{color}
\]

\subsection{defaultExpSettings}\label{Sec:SettingsExpM}

The Maple procedure defaultExpSettings constructs the Maple record
containing the default configuration settings for a polynomial-exponential system.

\[
\hbox{Settings := defaultExpSettings();}
\]

\begin{itemize}

\item {\tt Settings}

A Maple record containing the default configuration settings.
The fields in this record are:

\begin{itemize}
  \item {\tt algorithm}
  \item {\tt arithmeticType}
  \item {\tt precision}
  \item {\tt refineDigits}
  \item {\tt numRandomSystems}
  \item {\tt randomDigits}
  \item {\tt randomSeed}
  \item {\tt newtonOnly}
  \item {\tt numIterations}
  \item {\tt realityCheck}
  \item {\tt realityTest}
  \item {\tt deleteFiles}
\end{itemize}
\end{itemize}

\subsubsection{Example}

The following constructs the Maple record for the
configurations settings described in Table~\ref{Tab:Configs}.

\[
\begin{color}{red}\begin{array}{l}
> \hbox{Settings := defaultExpSettings();} \\
> \hbox{Settings:-algorithm := 0;} \\
> \hbox{Settings:-arithmeticType := 1;} \\
> \hbox{Settings:-precision := 192;}
\end{array}\end{color}
\]

\subsection{defaultSettings}\label{Sec:SettingsM}

The Maple procedure defaultSettings constructs the Maple record
containing the default configuration settings.

\[
\hbox{Settings := defaultSettings();}
\]

\begin{itemize}

\item {\tt Settings}

A Maple record containing the default configuration settings.
The fields in this record are:

\begin{itemize}
  \item {\tt algorithm}
  \item {\tt arithmeticType}
  \item {\tt precision}
  \item {\tt refineDigits}
  \item {\tt numRandomSystems}
  \item {\tt randomDigits}
  \item {\tt randomSeed}
  \item {\tt newtonOnly}
  \item {\tt numIterations}
  \item {\tt realityCheck}
  \item {\tt realityTest}
  \item {\tt deleteFiles}
\end{itemize}
\end{itemize}

\subsubsection{Example}

The following constructs the Maple record for the
configurations settings described in Table~\ref{Tab:Configs}.

\[
\begin{color}{red}\begin{array}{l}
> \hbox{Settings := defaultSettings();} \\
> \hbox{Settings:-algorithm := 0;} \\
> \hbox{Settings:-arithmeticType := 1;} \\
> \hbox{Settings:-precision := 192;}
\end{array}\end{color}
\]

\subsection{loadOutput}\label{Sec:OutputM}

The Maple procedure loadOutput loads output data.

\[
\hbox{PointsData := loadOutput(newtonOnly, algorithm, isReal, numVars);}
\]

\begin{itemize}

\item {\tt newtonOnly}

The value, either 0 or 1, of the configuration setting NEWTONONLY.

\item {\tt algorithm}

The value, either 0, 1, or 2, of the configuration setting ALGORITHM.

\item {\tt isReal}

A boolean value, i.e., either {\em true} or {\em false}, that describes
if the input polynomial system is a real polynomial system.

\item {\tt numVars}

The number of variables for the polynomial system.

\item {\tt PointsData}

A Maple record containing output data from \blueharvest.
See Section~\ref{Sec:alphaM} for the structure of this record.

\end{itemize}

\subsubsection{Example}

\[
\begin{color}{red}\begin{array}{l}
> \hbox{newtonOnly := 0;}\\
> \hbox{algorithm := 1;}\\
> \hbox{isReal := true;}\\
> \hbox{numVars := 2;}\\
> \hbox{PointsData := loadOutput(newtonOnly, algorithm, isReal, numVars);}
\end{array}\end{color}
\]

\subsection{printPoints}\label{Sec:PointsM}

The Maple procedure printPoints constructs an input point file.

\[
\hbox{printPoints(pointsName, Points, arithmeticType);}
\]

\begin{itemize}

\item {\tt pointsName}

A string which is the name of the file to create.

\item {\tt Points}

A two-dimensional list containing the points to print.

\item {\tt arithmeticType}

The value, either 0 or 1, of the configuration setting ARITHMETICTYPE.
If 0, the coordinates of the points are printed using a rational representation,
otherwise, the points are printed using a floating point representation.

\end{itemize}

\subsubsection{Example}

\[
\begin{color}{red}\begin{array}{l}
> \hbox{pointsName := ``testPoints'';} \\
> \hbox{Points := [[0,0],[1,1+I],[1,-I]];} \\
> \hbox{arithmeticType := 1;} \\
> \hbox{printPoints(pointsName, Points, arithmeticType);}
\end{array}\end{color}
\]

\subsection{printPolyExpSystem}\label{Sec:PolyExpM}

The Maple procedure printPolyExpSystem constructs a polynomial-exponential system file.

\[
\hbox{printPolyExpSystem(funcName, Funcs, Vars);}
\]

\begin{itemize}

\item {\tt funcName}

A string which is the name of the file to create.

\item {\tt Funcs}

A list containing the polynomial-exponential system.

\item {\tt Vars}

A list containing the variables of the polynomial-exponential system.

\end{itemize}

\subsubsection{Example}

\[
\begin{color}{red}\begin{array}{l}
> \hbox{funcName := ``polyExpSystem'';} \\
> \hbox{Funcs := [$x*y^2 + x, y - \exp(3*x)$];} \\
> \hbox{Vars := [$x,y$];} \\
> \hbox{printPolyExpSystem(funcName, Funcs, Vars);}
\end{array}\end{color}
\]

\subsection{printPolynomialSystem}\label{Sec:PolysM}

The Maple procedure printPolynomialSystem constructs a polynomial system file.

\[
\hbox{printPolynomialSystem(polyName, Polys, Vars);}
\]

\begin{itemize}

\item {\tt polyName}

A string which is the name of the file to create.

\item {\tt Polys}

A list containing the polynomial system.

\item {\tt Vars}

A list containing the variables of the polynomial system.

\end{itemize}

\subsubsection{Example}

\[
\begin{color}{red}\begin{array}{l}
> \hbox{polyName := ``polynomialSystem'';} \\
> \hbox{Polys := [$x*y^2 + x, x*y - y$];} \\
> \hbox{Vars := [$x,y$];} \\
> \hbox{printPolynomialSystem(polyName, Polys, Vars);}
\end{array}\end{color}
\]

\subsection{printSettings}\label{Sec:printSettingsM}

The Maple procedure printSettings constructs a configuration settings file.

\[
\hbox{printSettings(settingsName, Settings);}
\]

\begin{itemize}

\item {\tt settingsName}

A string which is the name of the file to create.

\item {\tt Settings}

A Maple record containing the configuration settings.
See Section~\ref{Sec:SettingsM} for more details constructing
this record.

\end{itemize}

\subsubsection{Example}

The following constructs the
configurations settings file described in Table~\ref{Tab:Configs}.

\[
\begin{color}{red}\begin{array}{l}
> \hbox{settingsName := ``settings'';} \\
> \hbox{Settings := defaultSettings();} \\
> \hbox{Settings:-algorithm := 0;} \\
> \hbox{Settings:-arithmeticType := 1;} \\
> \hbox{Settings:-precision := 192;} \\
> \hbox{printSettings(settingsName, Settings);}
\end{array}\end{color}
\]

\begin{thebibliography}{1}
\addcontentsline{toc}{chapter}{Bibliography}

\bibitem{Bertini}
D.J. Bates, J.D. Hauenstein, A.J. Sommese, and C.W. Wampler.
\newblock Bertini: Software for numerical algebraic geometry.
\newblock Available at \url{www.nd.edu/\~sommese/bertini}.

\bibitem{MPFR}
L. Fousse, G. Hanrot, V. Lef\`evre, P. P\'elissier, and P. Zimmermann.
\newblock {MPFR}: A Multiple-Precision Binary Floating-Point Library with Correct Rounding.
\newblock {\em ACM Trans. Math. Softw.}, 33(2), Art. 13, 2007.

\bibitem{GMP}
T. Granlund.
\newblock {GNU MP}: the GNU multiple precision arithmetic library.
\newblock Available at \url{www.gmplib.org}.

\bibitem{HS10}
J.D. Hauenstein and F. Sottile.
\newblock alphaCertified: certifying solutions to polynomial systems.
\newblock To appear in {\em ACM Trans. Math. Softw.}

\bibitem{HL11}
J.D. Hauenstein and V. Levandovskyy.
\newblock Certifying solutions to square systems of polynomial-exponential equations.
\newblock Preprint, 2011. Available at \url{www.math.tamu.edu/\~jhauenst/preprints}.


\end{thebibliography}

\appendix
\chapter{Configurations}\label{CHAP:configs}

\vskip -0.2in
The configurations for \blueharvestS are presented below
along with a brief description.

\begin{table}[h!]\label{good_settings}
\centering
\caption{Configurations for \blueharvest}
\begin{tabular}{|c|c|c|}
\hline
NAME & ACCEPTABLE VALUES & DEFAULT VALUE\\
\hline
\hline
ALGORITHM & 0, 1, 2 & 2\\
\hline
ARITHMETICTYPE & 0 or 1 & 0\\
\hline
PRECISION & $\geq$64 & 96\\
\hline
REFINEDIGITS & $\geq$0 & 0 \\
\hline
NUMRANDOMSYSTEMS & $\geq$2 & 2 \\
\hline
RANDOMDIGITS & $>0$ & 10 \\
\hline
RANDOMSEED & $>$0 & random \\
\hline
NEWTONONLY & 0 or 1 & 0 \\
\hline
NUMITERATIONS & $>0$ & 2 \\
\hline
REALITYCHECK & -1, 0, or 1 & 1 \\
\hline
REALITYTEST & 0 or 1 & 0 \\
\hline
\end{tabular}
\end{table}

\vskip 0.1in
\noindent {\bf \Large Configurations}
\begin{itemize}

\item ALGORITHM

If ALGORITHM is 0, \blueharvestS only determines which points are certifiably approximate solutions for the given polynomial system.
If ALGORITHM is 1, \blueharvestS also determines which certifiable approximate solutions correspond to distinct solutions.
If ALGORITHM is 2 and the polynomial system is real, i.e., has only real coefficients, \blueharvestS
also determines which certifiable approximate solutions correspond to real solutions.

\item ARITHMETICTYPE

If ARITHMETICTYPE is 0, \blueharvestS performs all computations using rational (certifiable) arithmetic.
If ARITHMETICTYPE is 1, \blueharvestS performs all computations using floating point arithmetic.
In this case, the results of \blueharvestS are {\it soft certified} since the floating point errors
are not fully controlled.  One way to control local errors is to increase PRECISION.

\item PRECISION

If ARITHMETICTYPE is 1, PRECISION indicates the starting level of precision (in bits).
That is, all computations for each point start with this precision, but the internal working precision
can be increased as needed.
Standard settings include 64 bits (roughly 19 decimal digits), 96 (28), 128 (38), 160 (48), 192 (57), 224 (67), and 256 (77).
In general, $N$ bits is equivalent to $\left\lfloor N\hbox{log}_{10}(2)\right\rfloor$ decimal digits.

\item REFINEDIGITS

If REFINEDIGITS is positive, say $\tau$, all of the certifiable approximate solutions will be refined
using Newton's method to be within $10^{-\tau}$ of the corresponding solution.  If ARITHMETICTYPE is 1,
the precision will automatically be increased internally so that the refined point is computed using
a precision that has at least $\tau$ decimal digits.

\item NUMRANDOMSYSTEMS

When the polynomial system is overdetermined, \blueharvestS will analyze NUMRANDOMSYSTEMS
number of randomized square systems.

\item RANDOMDIGITS

When the polynomial system is overdetermined, each randomized square system must
have a solution within $10^{-\tau}$ to be considered an approximate solution
for the overdetermined system, where $\tau$ is RANDOMDIGITS.

\item RANDOMSEED

RANDOMSEED is the seed for the random number generator.

\item NEWTONONLY

If NEWTONONLY is 1, \blueharvestS performs NUMITERATIONS number of Newton iterations on the points.
If ARITHMETICTYPE is 1, the precision will automatically be increased internally following each iteration.

\item NUMITERATIONS

The number of Newton iterations to perform when NEWTONONLY is 1, that is, when \blueharvestS is
only setup to only perform Newton iterations on the points.

\item REALITYCHECK

If REALITYCHECK is -1, \blueharvestS assumes that the polynomial system defines a real map.
Otherwise, the value of REALITYCHECK instructs \blueharvestS on which tests to perform
to determine if the polynomial system defines a real map
If REALITYCHECK is 0, \blueharvestS only checks to see if all of the coefficients
are real.  If REALITYCHECK is 1, \blueharvestS checks the coefficients as well as
checking to see if the polynomial system, as a set, is invariant under conjugation.

\item REALITYTEST

If REALITYTEST is 0, the local approach for determining reality of associated solutions
presented in \cite{HS10} is used.  If REALITYTEST is 1, the global approach presented
in \cite{HS10} is used.

\end{itemize}

\end{document}


